\documentclass[zh]{resume}

\setCJKmonofont{等距更纱黑体 SC}[
  AutoFakeBold=true,
  AutoFakeSlant=true
]

\special{dvipdfmx:config z 0} % 取消 PDF 压缩,加快速度,最终版本生成时注释该行

% File information shown at the footer of the last page
\fileinfo{%
  \faCopyright{} 2025, Haoran Guo \hspace{0.5em}
  \creativecommons{by}{4.0} \hspace{0.5em}
  \githublink{GuanghereGuo}{mycode} \hspace{0.5em}
  \faEdit{} \today
}

\name{浩然}{郭}

\profile{
  \mobile{180-1537-6195}
  \email{guanghere@mail.nwpu.edu.cn}
  \github{GuanghereGuo}
  \university{西北工业大学 \textbullet 民航学院}
  \degree{飞行器控制与信息工程 \textbullet 大一在读本科生}
  \birthday{2006.02}
  \address{无锡}
}

\photo[shape=rectangular, position=right, width=3cm, height=4.36cm]{./figures/1.jpg}

\begin{document}
\makeheader

%======================================================================
% Summary & Objectives
%======================================================================
\begin{abstract}
\textbf{西北工业大学飞行器控制与信息工程专业2024级本科生},GPA 3.967/4.1,具备扎实的数学基础和算法能力。
\end{abstract}

%======================================================================
\sectionTitle{教育背景}{\faGraduationCap}
%======================================================================
\begin{educations}
  \education%
    {2024.09}%
    [至今]%
    {西北工业大学}%
    {民航学院}%
    {飞行器控制与信息工程}%
    {本科在读}
\end{educations}

%======================================================================
\sectionTitle{学业水平}{\faBook}
%======================================================================
\begin{itemize}
  \item 截至\today,GPA达到 \(3.967/4.1\)
  \item 核心课程:微积分、线性代数、大学物理、电路基础、大学英语等课程 \(4.1/4.1\)
  \item 英语能力:CET-4成绩暂未公布,CET-4模拟成绩约640分(基于2024学年秋学期大学英语(高阶)与CET-4难度、题型相同的期末考试)
\end{itemize}

%======================================================================
\sectionTitle{计算机技能}{\faCode}
%======================================================================
\begin{itemize}
  \item \textbf{编程语言}:
    \begin{itemize}
      \item C/C++:掌握STL容器、基础OOP编程,作为ACM竞赛主力语言
      \item Python:了解NumPy、Matplotlib和Scipy,完成简单数据可视化和科学计算
    \end{itemize}
  \item \textbf{算法能力}:熟悉各种高级数据结构(如线段树、平衡树、并查集等),掌握常用算法与技巧(如动态规划、搜索算法、图论算法等),并能够将这些算法结合到实际问题中
  \item \textbf{工具}:熟悉Git版本控制基础操作,掌握Markdown标记语言,\LaTeX 初步入门
  \item \textbf{AI工具}:掌握与AI相结合的学习技巧,能够利用AI加强自己的学习效率和知识水平
\end{itemize}

%======================================================================
\sectionTitle{学科竞赛}{\faAtom}
%======================================================================
\begin{itemize}
  \item \textbf{程序设计竞赛}:
    \begin{itemize}
      \item \textbf{ACM-ICPC国际大学生程序设计竞赛}:
        \begin{itemize}
          \item 第49届ACM-ICPC国际大学生程序设计竞赛亚洲区域赛:rank 256/365
          \item 2025陕西省赛:银奖(解题数7/13)
          \item 2025西北工业大学春季校赛:\textbf{亚军}
        \end{itemize}
      \item 2025团体程序设计天梯赛:全国总决赛珠峰争顶组三等奖
      \item 参加THUPC2025清华大学程序设计竞赛暨高校邀请赛(2025)
    \end{itemize}
  \item \textbf{英语竞赛}
    \begin{itemize}
      \item 2024“外研社·国才杯”:英语组综合能力铜奖
      \item 2025全国大学生英语竞赛:二等奖
    \end{itemize}
  \item 西北工业大学数学建模竞赛二等奖
\end{itemize}

%======================================================================
\sectionTitle{实践探索}{\faFlask}
%======================================================================
\begin{itemize}
  \item 基于EasyX图形库实现一个简单的2048小游戏
  \item 基于开源项目\href{https://github.com/lobehub/lobe-chat}{LobechatDB}在云服务器部署个人AI智能助理,构建个人知识库,实现知识问答等功能以辅助学习
\end{itemize}

\end{document}